\section{Literature Review}

The basic idea from accounting. In the mid 2000s, several scholars began presenting work at conferences that suggested using tests based on Benford's law to detect election fraud \parencite{Mebane2006,Pericchi2004}.  Although some of these works were ultimately published in peer-reviewed journals \parencite{Pericchi2011}, in the early part of this decade there was a sense among some political scientists that those interested in detecting election fraud were getting ahead of themselves.  \textcite{Deckert2011} exemplifies this view:

\begin{quote}
If we judge things by the proliferation of Web sites and Internet blogs, the application of Benford’s Law to elections is now an important part of political science’s public face. Unfortunately, the ``research" offered is anything but peer reviewed. This essay, then, can be interpreted as an assessment of the conclusions that might apply were peer review a part of this component of the discipline’s public persona.
\end{quote}

Responding to \textcite{Deckert2011}, \textcite{Mebane2011} argued that his conference papers had been misinterpreted, in particular that he had always maintained that ``the 2BL test cannot detect all kinds of fraud, and significant
2BL test results may occur even when vote counts are in no way fraudulent."  Indeed, there have been a number of single country studies that demonstrate an alarming false discovery rate when using the second-Benford law alone. For example, \textcite{Shikano2011} analyze the 2009 German Federal Parliamentary election by using a chi-squared test against the second Benford law. For one major party, the percentage of constituencies whose election results were apparently suspicious while using a 5\%-level test was 22.4\%. \textcite{Breunig2011} finds violations in 51 out of 190 state-level elections in earlier German Parliamentary Elections. \textcite{Mebane2008} finds a surprising amount of fraud in various American counties without any reports of significant fraud.  On the other hand, \textcite{Beber2012} find that the last digits in Swedish Parliamentary elections in 2002 were consistent with a uniform distirbution, quite close to Benford's third-digit law. The latter paper argues such tests should be applied to the data.




 distribution to study election returns to see if they had been faked, were the first to propose applying such tests to   Transportation to Carter Center 2005, Shikano and Mack 2009, Lopez 2009, Deckert Myagkov and ORdeshook 2011, Mebane 2011, 2014 2013 .  Pericchi and Torres 2011 Cantu Saiegh 2011. Rundlett and Svolik 2015. Shpilkin 2011, Mebane and Kalinin 2009a,b